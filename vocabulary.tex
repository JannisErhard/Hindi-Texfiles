\subsection{Adpositions}
\begin{table}[H]
    \centering
    \begin{tabular}{c|c|c}        
    \begin{hindi}से \end{hindi} & from & se \\
    \begin{hindi}में \end{hindi} & in,within,during & me \\
    \begin{hindi}पर \end{hindi} & on,at & par \\
    \begin{hindi}का \end{hindi} & "'s" & ka \\
    \begin{hindi} पास \end{hindi} & nearby/belonging & pas \\
    \end{tabular}
    \caption{postpositions, ka indicates possession and can be related to the english " 's ". Pas indicates ownership and is in combinations with forms of to be translated as "has/have"}
    \label{tab:postpositions}
\end{table}

\newpage 
\subsection{Nouns}
\begin{table}[H]
    \centering
    \begin{tabular}{c|c|c}
        \begin{hindi} खाना \end{hindi} & food & khanaa \\
        \begin{hindi} फल \end{hindi} & fruits & Fhal \\ 
        \begin{hindi} केला \end{hindi} & banana & kela \\
        \begin{hindi} सेब \end{hindi} & apple & sehb \\
        \begin{hindi} नींबू \end{hindi} & lemon & nimboo \\
        \begin{hindi} संतर \end{hindi} & orange & santara \\
        \begin{hindi} आम \end{hindi} & mango & aam \\ 
        \begin{hindi} सब्ज़ी \end{hindi} & vegetable & sabzee \\ 
        \begin{hindi} टमाटर \end{hindi} & tomatoe & tamatar \\
        \begin{hindi} आलू \end{hindi} & potatoe & alu \\
        \begin{hindi} चावल \end{hindi} & rice & Chavaal \\
        \begin{hindi} गाजर \end{hindi} & carrot & gajar \\ 
        \begin{hindi} अंडा \end{hindi} & egg & anda \\
        \begin{hindi} मछली \end{hindi} & fish & manchalee \\  
        \begin{hindi} पनीर \end{hindi} & cheese & paneer \\  
        \begin{hindi} मक्खन \end{hindi} & butter & makhan \\  
        \begin{hindi} पानी \end{hindi} & water & pani \\
        \begin{hindi} चाय \end{hindi} & tea & chai \\
        \begin{hindi} शराब \end{hindi} & alcohol & Sharaab \\

    \end{tabular}
    \caption{Food}
    \label{tab:nouns_food}
\end{table}

\begin{table}[H]
    \centering
    \begin{tabular}{c|c|c}
        \begin{hindi} सवाल \end{hindi} & question & savaal \\
        \begin{hindi} जवाब \end{hindi} & answer & javaab \\ 
        \begin{hindi}गाना \end{hindi} & sing & gaana \\ 
    \end{tabular}
    \caption{communication}
    \label{tab:nouns_communication}
\end{table}

\begin{table}[H]
    \centering
    \begin{tabular}{c|c|c}
        \begin{hindi} दर्द \end{hindi} & pain & dard \\  
        \begin{hindi} बुखार \end{hindi} & fever & bukhar \\  
        \begin{hindi} दवाई \end{hindi} & medicine & davai \\ 
    \end{tabular}
    \caption{Illness}
    \label{tab:nouns_illness}
\end{table}

\begin{table}[H]
    \centering
    \begin{tabular}{c|c|c}
\begin{hindi} समय \end{hindi} & time & sahmeh \\  
\begin{hindi} मिनट \end{hindi} & minute & minute \\  
\begin{hindi} घंटा \end{hindi} & hour & ghanta \\ 
\begin{hindi} दिन \end{hindi} & day & day \\ 
\begin{hindi} हफ्ता \end{hindi} & week & haphta \\ 
\begin{hindi} महीना \end{hindi} & month & maheena \\ 
\begin{hindi} साल \end{hindi} & year & saal \\ 
\begin{hindi} शाम \end{hindi} & evening & sham \\ 
\begin{hindi} दोपहर \end{hindi} & afternoon & dopeher \\ 
\begin{hindi} सुबह \end{hindi} & morning & suhbah \\ 
\begin{hindi} रात \end{hindi} & night & rhad \\ 
\begin{hindi} डेढ़ \end{hindi} & half past one & dedh \\ 
\begin{hindi} ढाई \end{hindi} & half past two & dhaee \\ 
    \end{tabular}
    \caption{Time}
    \label{tab:nouns_time}
\end{table}

\begin{table}[H]
    \centering
    \begin{tabular}{c|c|c}
\begin{hindi} सोमवार \end{hindi} & monday & somavaar \\  
\begin{hindi} मंगलवार \end{hindi} & thuesday & mangalavaar \\  
\begin{hindi} बुधवार \end{hindi} & wednesday & budhavaar \\ 
\begin{hindi} गुरुवार \end{hindi} & thursday & guruvaar \\ 
\begin{hindi} शुक्रवार \end{hindi} & friday & shukravaar \\ 
\begin{hindi} शनिवार \end{hindi} & saturday & shanivaar \\ 
\begin{hindi} रविवार \end{hindi} & sunday & ravivaar \\ 
    \end{tabular}
    \caption{Days of the Week}
    \label{tab:nouns_weekdays}
\end{table}

\begin{table}[H]
    \centering
    \begin{tabular}{c|c|c}
\begin{hindi} जनवरी \end{hindi} & january & janavaree \\  
    \end{tabular}
    \caption{The months}
    \label{tab:nouns_months}
\end{table}

\begin{table}[H]
    \centering
    \begin{tabular}{c|c|c}
\begin{hindi} हाथ \end{hindi} & hand & hat \\ 
\begin{hindi} पैर \end{hindi} & foot & pair \\ 
\begin{hindi} नाक \end{hindi} & nose & nak \\ 
\begin{hindi} कान \end{hindi} & ear & kan \\ 
\begin{hindi} मुँह \end{hindi} & mouth & muhn \\ 
\begin{hindi} बाल \end{hindi} & hair & bal \\ 
\begin{hindi} सिर \end{hindi} & head & sir \\ 
\begin{hindi} अँगूठा \end{hindi} & thumb & angootha \\ 
\begin{hindi} अंगुली \end{hindi} & finger & angulee \\ 
\begin{hindi} दिल \end{hindi} & heart & dil \\ 
\begin{hindi} दाँत \end{hindi} & teeth & dant \\ 
\begin{hindi} आँख \end{hindi} & eye & aankh \\  
    \end{tabular}
    \caption{Basic Anatomy}
    \label{tab:nouns_anatomy}
\end{table}


\begin{table}[H]
    \centering
    \begin{tabular}{c|c|c}
        \begin{hindi} बेटा \end{hindi} & son & beta \\
        \begin{hindi} बेटी \end{hindi} & daughter & beti\\
        \begin{hindi} भाई \end{hindi} & brother & bhai\\
        \begin{hindi} बहन \end{hindi} & sister & behen\\
        \begin{hindi} माँ \end{hindi} & mother & mah\\
        \begin{hindi} पिता \end{hindi} & father & pitta\\
        \begin{hindi} दादी \end{hindi} & paternal grandmother & dadi\\
        \begin{hindi} दादा \end{hindi} & paternal grandfather & dada\\
        \begin{hindi} नानी  \end{hindi} & maternal grandmother & nani \\
        \begin{hindi} नाना \end{hindi} & maternal grandfather & nana \\
        \begin{hindi} चाची \end{hindi} & aunt & chaachee \\
        \begin{hindi} चाचा \end{hindi} & uncle & chaacha \\
        \begin{hindi} परिवार \end{hindi} & family & parivar\\
        \begin{hindi} पति \end{hindi} & husband & pati\\
        \begin{hindi} पत्नी \end{hindi} & wife  & patnee\\
        \begin{hindi} रिश्तेदार \end{hindi} & relative  & rishtedaar\\
        \begin{hindi} रिश्ता \end{hindi} & relationship  & rishta\\
    \end{tabular}
    \caption{Family}
    \label{tab:nouns_family}
\end{table}



\begin{table}[H]
    \centering
    \begin{tabular}{c|c|c}
        \begin{hindi} लोग \end{hindi} & people & log \\
        \begin{hindi} दोस्त \end{hindi} & friend & dhost\\
        \begin{hindi} आदमी \end{hindi} & man & admee\\
        \begin{hindi} औरत \end{hindi} & woman & aurat\\
        \begin{hindi} लड़का \end{hindi} & boy & ladakaa\\
        \begin{hindi} लड़की \end{hindi} & girl & ladakee\\
        \begin{hindi} बच्चा \end{hindi} & child & batscha \\
        \begin{hindi} पड़ोसी \end{hindi} & neighbour & padosee \\
    \end{tabular}
    \caption{People}
    \label{tab:nouns_people}
\end{table}


 \begin{table}[H]
    \centering 
    \begin{tabular}{c|c|c}
        \begin{hindi} पक्षी \end{hindi} & bird & pepshi \\
        \begin{hindi} कबूतर \end{hindi} & pidgeon & kabuter \\
        \begin{hindi}  मोर \end{hindi} & peacock & mohr \\
        \begin{hindi} बिल्ली \end{hindi} & cat & billi \\
        \begin{hindi}  गाय \end{hindi} & cow & gai \\
        \begin{hindi}  घोड़ा  \end{hindi} & horse& ghora  \\
        \begin{hindi}  कुत्ता \end{hindi} & dog & kutta \\
        \begin{hindi}  हाथी \end{hindi} & elephant & hathi \\
        \begin{hindi}  चूहा \end{hindi} & mouse & chooha \\
        \begin{hindi}  जानवर \end{hindi} & animal & dschahnvar \\
    \end{tabular}
    \caption{animals}
    \label{tab:nouns_animals}
\end{table}
 
 \begin{table}[H]
    \centering 
    \begin{tabular}{c|c|c}
        \begin{hindi} लेखक \end{hindi} & writer &  lekhak\\
        \begin{hindi} शिक्षक  \end{hindi} &  teacher & shikshak \\
    \end{tabular}
    \caption{professions}
    \label{tab:nouns_professions}
\end{table}


\begin{table}[H]
    \centering 
    \begin{tabular}{c|c|c}
        \begin{hindi} स्कूल \end{hindi} & school & skul \\
        \begin{hindi} घर \end{hindi} & home & Ghr \\
        \begin{hindi} भारत \end{hindi} & indien & bharatt \\
        \begin{hindi} सड़क \end{hindi} & road & saddak \\
        \begin{hindi} नदी \end{hindi} & river & nadhee \\  
        \begin{hindi} गाँव \end{hindi} & village & ghav \\  
        \begin{hindi} शहर \end{hindi} & city & sheher \\  
        \begin{hindi}   गली\end{hindi} & street & galee \\  
   \begin{hindi}   जगह \end{hindi} & place & jagah \\  
    \end{tabular}
    \caption{places}
    \label{tab:nouns_places}
\end{table}

\begin{table}[H]
    \centering 
    \begin{tabular}{c|c|c}
        \begin{hindi} कुर्सी \end{hindi} & chair & kursi \\
        \begin{hindi} मेज़ \end{hindi} & table & mehz \\
        \begin{hindi} पैसा \end{hindi} & money & paisa \\
        \begin{hindi} रुपया \end{hindi} & rupee & rupia \\
        \begin{hindi} गाड़ी \end{hindi} & vehicle & gaadee \\
        \begin{hindi} मकान \end{hindi} & house & makaan \\
        \begin{hindi} दरवाज़ा \end{hindi} & door & darwaza \\
        \begin{hindi} दीवार \end{hindi} & wall & deevaar \\
        \begin{hindi} खिड़की \end{hindi} & window & keerkee \\
        \begin{hindi} ज़मीन \end{hindi} & floor & zameen \\
        \begin{hindi} छत \end{hindi} & roof & chat \\
        \begin{hindi} बिस्तर \end{hindi} & bed & bistar \\
        \begin{hindi} कमरा \end{hindi} & room & kamara \\
    \end{tabular}
    \caption{inanimate objects}
    \label{tab:nouns_inanimate}
\end{table}

\begin{table}[H]
    \centering 
    \begin{tabular}{c|c|c}
            \begin{hindi} कपड़े \end{hindi} & clothes & kapade \\
        \begin{hindi} जुर्राब \end{hindi} & sock & jurraab \\
        \begin{hindi} जूता \end{hindi} & shoe & joota \\
        \begin{hindi} कमीज \end{hindi} & shirt & kameej \\
        \begin{hindi} पतलून \end{hindi} & trousers & pataloon \\
                \begin{hindi} छाता \end{hindi} & umbrella & chhaata \\
    \end{tabular}
    \caption{clothing}
    \label{tab:nouns_clothing}
\end{table}




\begin{table}[H]
    \centering 
    \begin{tabular}{c|c|c}
        \begin{hindi} काम \end{hindi} & work & kam \\
    \end{tabular}
    \caption{abstract objects}
    \label{tab:nouns_abstract}
\end{table}

\begin{table}[H]
    \centering
    \begin{tabular}{c|c|c}        
    \begin{hindi} एक \end{hindi} & one & ek \\
    \begin{hindi} दो  \end{hindi} & two & do \\
    \begin{hindi} तीन \end{hindi} & three & tin \\
    \begin{hindi} चार  \end{hindi} & four & chaar \\
    \begin{hindi} पाँच \end{hindi} & five & panch \\
    \begin{hindi} छह \end{hindi} & six & cheh \\
    \begin{hindi} सात \end{hindi} & seven & saat  \\
    \begin{hindi} आठ \end{hindi} & eight & aath \\
    \begin{hindi} नौ  \end{hindi} & nine & nau \\
    \begin{hindi} दस  \end{hindi} & ten & das \\
    \begin{hindi} सौ  \end{hindi} & hundered & sau \\
    \begin{hindi} हज़ार  \end{hindi} & thousand & hazaar \\
    \begin{hindi} कई  \end{hindi} & many & kaee \\
    \begin{hindi} आधा  \end{hindi} & half & adha \\
    \begin{hindi} कुछ  \end{hindi} & some & kutsh \\
    \end{tabular}
    \caption{Cardinals, see the corresponding ordinals in Table \ref{tab:adjective_ordinals}}
    \label{tab:nouns_cardinals}
\end{table}

\begin{table}[H]
    \centering
    \begin{tabular}{c|c|c}        
    \begin{hindi} अखबार\end{hindi} & newspaper 	& akhabaar \\
    \begin{hindi}चिट्ठी  \end{hindi} & letter 	&  citthi\\
    \begin{hindi}किताब   \end{hindi} & book 	&  kitaab\\
    \begin{hindi}कविता   \end{hindi} & poem 	&  kavita\\
    \begin{hindi}कहानी   \end{hindi} & story	&  kahaanee \\
    \end{tabular}
	\caption{nouns describing forms of the written word}
    \label{tab:nouns_literature}
\end{table}

\newpage 
\subsection{Verbs}


\begin{table}[H]
    \centering
    \begin{tabular}{c|c|c}
        \begin{hindi} पढ़ना \end{hindi} & read &  padhana\\
        \begin{hindi} पीना \end{hindi} & drink &  pina \\
        \begin{hindi} खाना \end{hindi} & eat &  khana \\
        \begin{hindi} रहना \end{hindi} & stay &  rhana \\
        \begin{hindi} करना \end{hindi} & work &  karana \\
        \begin{hindi} खेलना \end{hindi} & play &  khlana \\
        \begin{hindi} सोना \end{hindi} & sleep & sona \\
        \begin{hindi} जागना \end{hindi} & wake up & jāgtā \\
    \end{tabular}
    \caption{Infinitive Any}
    \label{tab:verbs_any}
\end{table}



\begin{table}[H]
    \centering
    \begin{tabular}{c|c|c}
        \begin{hindi} बातना \end{hindi} & talk & batna \\
        \begin{hindi} बोलना \end{hindi} & speak & bolna \\
        \begin{hindi} पूछना \end{hindi} & ask & poochna \\
    \end{tabular}
    \caption{Infinitive verbs describing communication}
    \label{tab:verbs_communication}
\end{table}


\begin{table}[H]
    \centering
    \begin{tabular}{c|c|c}
     \begin{hindi} बैठना  \end{hindi} & sit & beth \\
     \begin{hindi} जाना \end{hindi} & go & ja, jaata \\
     \begin{hindi} आना \end{hindi} & come & aa, aata \\
     \begin{hindi} उड़ना \end{hindi} & fly & ur, urta \\
     \begin{hindi} तैरना \end{hindi} & swim & ther, therta \\
     \begin{hindi} चलना \end{hindi} & walk & jhal, jhalta \\
     \begin{hindi} दौड़ना \end{hindi} & run & daur, dauta \\
    \end{tabular}
    \caption{Movement}
    \label{tab:verbs_move}
\end{table}



\begin{table}[H]
    \centering
    \begin{tabular}{c|c|c}
        \begin{hindi}चाहिए \end{hindi} & need, want & chaahie\\
        \begin{hindi}पसंद  \end{hindi} & like & pasad\\
    \end{tabular}
    \caption{emotions}
    \label{tab:verb_emotions}
\end{table}

\begin{table}[H]
    \centering
    \begin{tabular}{c|c|c}
        \begin{hindi} खेल \end{hindi} & game, plays & khel \\
    \end{tabular}
    \caption{Double meanings}
    \label{tab:doublemeanings}
\end{table}

\newpage  
\subsection{Adjectives}

\begin{table}[H]
    \centering
    \begin{tabular}{c|c|c}        
    \begin{hindi} खुश \end{hindi} & happy & kush \\
    \begin{hindi} दुखी \end{hindi} & sad & dukhee \\
    \end{tabular}
    \caption{emotions}
    \label{tab:adjective_emotions}
\end{table}

\begin{table}[H]
    \centering
    \begin{tabular}{c|c|c}        
    \begin{hindi} खड़ा \end{hindi} & standing & khara \\
    \end{tabular}
    \caption{any}
    \label{tab:adjective_any}
\end{table}


\begin{table}[H]
    \centering
    \begin{tabular}{c|c|c}        
    \begin{hindi} रंग \end{hindi} & color & rang \\
    \begin{hindi} काला \end{hindi} & black & kala \\
    \begin{hindi} सफ़ेद \end{hindi} & white & safed \\
    \begin{hindi} लाल\end{hindi} & red & lal \\
    \begin{hindi} हरा\end{hindi} & green & raha \\
    \begin{hindi} नीला\end{hindi} & blue & nila \\
    \begin{hindi} पीला\end{hindi} & yellow & pila \\
    \end{tabular}
    \caption{colors}
    \label{tab:adjective_colors}
\end{table}

\begin{table}[H]
    \centering
    \begin{tabular}{c|c|c}        
    \begin{hindi} लंबा  \end{hindi} & tall & lamba \\
    \begin{hindi} छोटा  \end{hindi} & short & chota \\
    \begin{hindi} पतला \end{hindi} & thin & patalla \\
    \begin{hindi} मोटा  \end{hindi} & fat & mota \\
    \begin{hindi} बूढ़ा  \end{hindi} & old & budha \\
    \begin{hindi} बड़ा  \end{hindi} & big & bada \\
    \begin{hindi} बुरा  \end{hindi} & bad & bura \\
    \begin{hindi} जवान \end{hindi} & young & javahn \\
    \begin{hindi} सुंदर   \end{hindi} & attractive & sunder \\
    \end{tabular}
    \caption{characteristics}
    \label{tab:adjective_colors}
\end{table}


\begin{table}[H]
    \centering
    \begin{tabular}{c|c|c}        
    \begin{hindi} पहला \end{hindi} & first & pahlee \\
    \begin{hindi} दूसरा  \end{hindi} & second & dosra \\
    \begin{hindi} तीसरा \end{hindi} & third & teesra \\
    \begin{hindi} चौथी  \end{hindi} & fourth & chautee \\
    \begin{hindi} पांचवां \end{hindi} & fifth & panchva \\
    \begin{hindi} छठा \end{hindi} & sixth & chatha \\
    \end{tabular}
    \caption{ordinals}
    \label{tab:adjective_ordinals}
\end{table}

\begin{table}[H]
    \centering
    \begin{tabular}{c|c|c}
        \begin{hindi} खुला \end{hindi} & open & kula \\
        \begin{hindi} बंद \end{hindi} & closed & band \\
        \begin{hindi} नया \end{hindi} & new & naee \\
        \begin{hindi} महँगा \end{hindi} & expensive & mahgaa \\
        \begin{hindi} सस्ता \end{hindi} & cheap & sasta \\
    \end{tabular}
    \caption{describing objects}
    \label{tab:adjectives_object_describing}
\end{table}

\begin{table}[H]
    \centering
    \begin{tabular}{c|c|c}
        \begin{hindi} ग़लत \end{hindi} & wrong & galat \\
        \begin{hindi} सही \end{hindi} & right & sahee \\
        \begin{hindi} आसान \end{hindi} &  easy & aasaan \\\
        \begin{hindi}मुश्किल \end{hindi} &  edifficult & muskila \\\
    \end{tabular}
    \caption{abstract properties}
    \label{tab:adjectives_abstract}
\end{table}

\begin{table}[H]
    \centering
    \begin{tabular}{c|c|c}
        \begin{hindi} साढ़े \end{hindi} & half past  & saadhe \\
        \begin{hindi} सवा \end{hindi} & quarter past  & sava \\
        \begin{hindi} पौने \end{hindi} & quarter to  & paune \\
    \end{tabular}
    \caption{temporal adjectives}
    \label{tab:adjectives_temporal}
\end{table}

\newpage
\subsection{Adverbs}

\begin{table}[H]
    \centering
    \begin{tabular}{c|c|c}
\begin{hindi} बजे \end{hindi} & o'clock & baje \\  
\begin{hindi} सुबह \end{hindi} & in the morning & subbah \\  
\begin{hindi} से \end{hindi} & since & se \\  
\begin{hindi} अब \end{hindi} & now & ab \\  
\begin{hindi} कब \end{hindi} & when & kab \\  
\begin{hindi} तब \end{hindi} & then & tab \\  
\begin{hindi} कभी \end{hindi} & ever & kabhee \\  
\begin{hindi} कभी-कभी \end{hindi} & sometimes & kabhee-kabhee \\  
\begin{hindi} कभी नहीं\end{hindi} & never & kabhee-nahee \\  
\begin{hindi} हमेशा\end{hindi} & always & hamesha \\  
\begin{hindi} बाद में\end{hindi} & later & baad mein \\ 
\begin{hindi} बाद \end{hindi} & after & baad \\   
    \end{tabular}
    \caption{Temporal adverbs}
    \label{tab:adverbs_temporal}
\end{table}


\begin{table}[H]
    \centering
    \begin{tabular}{c|c|c}        
    \begin{hindi} कैसे \end{hindi} & how & kaise \\
    \begin{hindi} कौन \end{hindi} & who & kaun \\
    \begin{hindi} कितने \end{hindi} & how many & kit(an)e \\
    \begin{hindi} कहाँ \end{hindi} & where & kahaa \\
    \begin{hindi} कब \end{hindi} & when & kab \\
    \end{tabular}    
    \caption{interrogative}
    \label{tab:adverbs_interrogative}
\end{table}

\begin{table}[H]
    \centering
    \begin{tabular}{c|c|c}        
    \begin{hindi} पीछे \end{hindi} & behind & pechee \\
    \begin{hindi} आगे \end{hindi} & ahead & aage  \\
    \begin{hindi} अंदर \end{hindi} & inside & andar \\
    \begin{hindi} बाहर \end{hindi} & outside & bahaar \\
    \begin{hindi} दूर \end{hindi} & far & duur \\
    \begin{hindi} ऊपर \end{hindi} & up & oopar \\
    \begin{hindi} के ऊपर\end{hindi} & on top of & oopar ke \\
    \begin{hindi} यहाँ \end{hindi} & here & yaha \\
    \begin{hindi} वहाँ \end{hindi} & there & vaha \\
    \end{tabular}    
    \caption{spatial}
    \label{tab:adverbs_spatial}
\end{table}




\newpage 
\subsection{Pronouns}
\begin{table}[H]
    \centering
    \begin{tabular}{c|c|c}
    \begin{hindi} वह \end{hindi} & that he/she & wah \\
    \begin{hindi} यह \end{hindi} & this she,it& yah \\
    \begin{hindi} वे \end{hindi} & they &  Weh \\
    \begin{hindi} मैं \end{hindi} & I & mäh \\    
    \begin{hindi} हम \end{hindi} &  we & ham \\    
    \begin{hindi} तुम \end{hindi} &  you (familiar) & Tum\\    
    \begin{hindi} आप \end{hindi} &  you (formal) & Aaap \\    
    \begin{hindi} उसका  \end{hindi} & their her,his  & Uska \\    
    \begin{hindi} मेरा \end{hindi} &  my & Mera \\    
    \begin{hindi} तेरा \end{hindi} &  your & Tera \\    
    \begin{hindi} उनके \end{hindi} & their & unke \\
    \begin{hindi} उनका \end{hindi} & her & Unka \\
    \begin{hindi} आप \end{hindi} & you & aap \\
    \end{tabular}
    \caption{pronouns}    
    \label{tab:pronouns}
\end{table}

\newpage
\subsection{phrases}
\begin{table}[H]
    \centering
    \begin{tabular}{c|c|c}
    \begin{hindi} से प्यार करते हैं \end{hindi} & to love someone & se pyaar karate hain \\    
    \begin{hindi} घर पर \end{hindi} & at the house & ghar par \\    
    \begin{hindi}  गर्मी का मौसम  \end{hindi} & summer season & garmee ka mausam \\    
    \begin{hindi}  सर्दी का मौसम  \end{hindi} & winter season & sardee ka mausam \\    
    \end{tabular}
    \caption{phrases}    
    \label{tab:phrases}
\end{table}



